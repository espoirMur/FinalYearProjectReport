\makeatletter
\renewenvironment{abstract}{%
	\if@twocolumn
	\section*{\abstractname}%
	\else %% <- here I've removed \small
	\begin{center}%
		{\bfseries\Large\textit {\abstractname}\vspace{\z@}}%  %% <- here I've added \Large
	\end{center}%
	\quotation
	\fi}
{\if@twocolumn\else\endquotation\fi}
\makeatother
\selectlanguage{french} 
\begin{abstract}
	\addcontentsline{toc}{chapter}{Résumé}
	\thispagestyle{plain} 
Aujourd'hui toutes les entreprises collectent et stockent de grandes quantités de données. Ces méga-bases de données, qui ne cessent d'augmenter jour après jour, sont peu exploitées, alors qu'elles cachent de connaissances décisives c'est ainsi qu'est né : le data mining   (qu'on appellerait en français fouille des données ).
\paragraph{}
Des nombreuses recherches ont été menées sur l'application du data mining dans différents domaines,  entre autre : les banques , telecom , assurances, etc
Mais peu des recherches de ce genre sont menées sur l'application du data mining dans le domaine de l'éducation. c'est ainsi que nous avons décider 
de nous pencher sur le sujet ayant pour thème \textit{ \textbf data mining à l’amélioration  de l'orientation des élèves finalistes du secondaire à l'université } .
\paragraph{}
Pour mener a bien ce travail nous nous sommes posé la question suivante \emph{Comment utiliser les techniques  du data mining pour doter les universités des outils d'aide à la décision les permettant de bien orienter les étudiants dès leur entrée à l'université ?  } et pour y répondre nous avons utilisé la méthodologie  \ac{CRISP-DM}:celle ci est la méthodologie la plus utilisée en industrie pour les projet data mining.
\paragraph{}
Nous avons travaillé sur un ensemble d'apprentissage issue du système d'information UAT de dimensions 9606 X 22 qui représentait les données pour chaque étudiant de l'ULPGL des années 2012-2016;sur base de ces données nous avons découvert la répartition des étudiants de l'ULPGL selon différents critères:  L'age de l'étudiant ,le pourcentage à l'exetat ,le sexe, l'école de provenance , l'option suivie, la nature de l'école  (nos inputs). 
En effectuant les statistiques inférentielles (Test khi-Carrée et Test ANOVA) nous avons remarqué que les étudiants choisissent leur faculté  en se basant sur ces inputs.
\paragraph{}
Ensuite nous avons créé une variable : le CGPA que nous avons utilisé pour évaluer la réussite d'un étudiant à l'université 
elle n'est rien d'autre que la moyenne des points obtenus par celui ci tout au long de son cursus académique.
Nous avons analyser la relation existant entre nos inputs et cette variable de sortie le CGPA au sein de chaque faculté  par la corrélation de Pearson , et le test ANOVA  et nous avons remarqué que le CGPA ne dépend pas de l'age , ni du sexe , et encore moins du pourcentage obtenu à l'exetat (coefficient de Pearson de 0.45) mais uniquement de l'école de provenance et de l'option suivie par l'étudiant à l'école secondaire.
\paragraph{}
Enfin , nous avons  décidé de prédire le CGPA au sein de chaque faculté pour un  étudiant  nouveau en se basant sur son pourcentage obtenu à l'exetat, l'école de provenance, et l'option suivie à l'école en utilisant  5 techniques du Machine Learning dont trois modèles de régression et deux techniques de \ac{SVM} .
Nous avons évaluer le modèle final en nous basant sur l'erreur moyen quadratique \ac{RMSE} et une validation croisée et avons obtenu un score de moins de 10\% pour presque toutes les facultés.\\
Notons que nous avons utilisé les librairies et frameworks  suivantes implémentées en python pour réaliser ce travail:  \textit {pandas, matplotlib,seaborn, flask  numpy et  scikit-learn }
\end{abstract}
\providecommand{\keywords}[1]{\textbf{\textit{Keywords  : }} #1}
\keywords{\textbf{\textit{
python,
machine-learning,
statistics,
data-mining,
education,
data-science,
educational-data-mining,
student-orientation	
}}
}