\chapter{Analyse du domaine de l'orientation  \cite{OrBk1} \cite{OrBk2} \cite{OrBk3} \cite{OrBk4} \cite{OrBk5}}
\section{Généralités sur l’orientation des étudiants}
\paragraph{}
\ac {UNAF}  considère l’orientation comme un  véritable parcours, qui comporte des étapes  certes, mais qui doit s’inscrire dans la durée.  Cette inscription dans la durée nécessite que  des passerelles multiplient entre les différentes filières afin de permettre les  réorientations, même en cours d’année. Ces  passerelles devraient par ailleurs permettre  aux jeunes d’aller aussi loin qu’ils le désirent  dans leurs études, quel que soit leur choix  initial. Ainsi l’orientation pourra être  dédramatisée et aucune formation choisie ne  sera une « voie sans issue ».  Ce serait aussi un moyen de lutter contre la  déscolarisation qui provient souvent d’un  choix d’orientation qui ne convient pas et qui  pousse les jeunes à arrêter complètement leur  scolarité au lieu de se réoriente
\paragraph{}
En effet, le terme “orientation” recouvre deux activités que la langue anglaise distingue : le processus qui répartit les élèves dans différentes voies de formation, filières et options (“students distribution”) ; l’aide aux individus dans le choix de leur avenir scolaire et professionnel (“vocational guidance”, “school and career counseling”).
\paragraph{}
Face à la crise, le diplôme protège du  chômage et favorise l’accès à la  formation continue.  Le rapport  « Formations et emplois » de l’Insee (2013), fait preuve qu’environ 800 000 élèves sont inscrits en  troisième et 700 000 sont candidats au  bac. Il s’avère important de savoir si chacun trouvera-t-il la place qui  lui convient pour la suite ? Ce n’est pas si facile, car la plupart des filières imposent une sélection à  l’entrée, en fonction du nombre de  places disponibles
\paragraph{}
La difficulté face à un choix d’orientation,  légitime à l’adolescence, peut être  handicapante par la suite, d’où la nécessité  d’une réelle  éducation au choix, à l’orientation qui prennent en compte les trois  volets : connaissance de soi, connaissance  des métiers et de l’environnement  professionnel, connaissance des formations. Cependant, au final, faute d’avoir trouvé sa voie ou acquis le niveau nécessaire, un nombre important d’élèves quitte chaque année le système éducatif sans avoir obtenu de qualification. 
\paragraph{}
A ce propos, le rapport  haut conseil d’éducation(2008) stipule que chaque année 120 000 jeunes quittent le  système de formation initiale sans  diplôme (surtout dans la voie  professionnelle). A l’université, le  décrochage concerne 80 000 jeunes par  an (1/4 de l’ensemble des sortants de  l’université d’une année). Les services qui s’occupent de  l’orientation sont éparpillés, il existe un accès inégal à ces services,  les ressources humaines et matérielles limitées influent sur la qualité des services offerts Pourtant, l’objectif des universités vise à offrir une bonne démarche d’orientation aux  étudiants. Elles ont comme défi d’élaborer  pour chaque étudiant une démarche d’orientation stimulante, cohérente, réaliste et adaptée à sa dynamique personnelle, d’établir une véritable culture d’orientation en impliquant, d’une part, tout le personnel de l’université, les amis et la famille de l’élève ainsi que les partenaires de la communauté et, d’autre part, en unissant les forces du réseau d’intervenants qui soutiennent les étudiants sur le plan personnel et professionnel pour faciliter leur insertion dans notre société
\paragraph{}
L’orientation au collège et au lycée est  déterminée par les notes qui répartissent  les élèves de manière hiérarchisée, et non  sur les appétences et potentialités  personnelles. (Au collège plus de 4 élèves  sur 10 estiment donc que leur orientation  est subie plus que choisie), elle peine à encourager les  jeunes issus des milieux les moins  favorisés, elle est trop irréversible c'est-à-dire absence de passerelles et de réorientation  entre voie professionnelle et voie  technologique et générale mais aussi elle est conditionnée par l’offre  de formation de proximité et cette offre  de formation est mal répartie sur le  territoire. On constate également la mobilité des élèves qui est faible, d’où  l’offre de formation ne s’adapte que  lentement aux nécessités économiques ce qui conduisent  les jeunes à s’engager  parfois dans des filières sans perspective. 
\paragraph{}
Ainsi, actuellement en \ac{RDC}, aucune structure d’orientation scolaire et professionnelle n’existe et le pouvoir public n’y pense même pas. Avec la prolifération des Instituts Supérieurs et des Universités publics ou privés, créées anarchiquement par le pouvoir public depuis des décennies, les élèves finalistes de l’enseignement secondaire s’orientent en considérant, en fin de compte, l’implantation géographique de ces établissements. Le système de prime qui a été instauré depuis les années 1992 complique encore davantage le système d’orientation des élèves. Les enfants issus des catégories sociales défavorisées embrassent, pour la plupart, les études dans les institutions publiques qui coûtent moins cher comparativement aux institutions privées. De ce fait, beaucoup d’élèves effectuent le choix des études aléatoirement sans tenir compte des aptitudes intellectuelles ou des aspirations professionnelles futures. Les jeunes sont donc abandonnés à leur triste sort. Cette situation étant considérée comme étant normale, personne n’en fait nullement allusion.
\section{Présentation du service de l’orientation des étudiants}
\textbf{\textit{Qui intervient dans le processus d’orientation ?}} \\

L’élève en quête d’orientation n’est pas laissé à lui-même.  Outre sa famille, il peut requérir l’aide de ses parents, de ses enseignants, du directeur d’établissement et des conseillers.
\begin{enumerate}
	\item \textit{\textbf{les parents: }} \\
	Parmi les acteurs de l’orientation, les parents sont bien entendu les premiers concernés. Si le rôle des parents est irremplaçable, c’est  avant tout parce qu’ils sont probablement seuls capable d’aimer l’enfant de manière inconditionnelle, indépendamment de ses résultats scolaires et autres performances. Ils sont les seuls à continuer à croire en l’enfant, lorsque les autres désespèrent. Et l’une des fonctions parentales est justement de construire l’estime de soi de leur enfant.
Ils interviennent pour trois objectifs qui sont :	
-  améliorer l’orientation afin qu’elle soit choisie plutôt que subie
-  renforcer les relations école-famille 
-  prévenir le décrochage scolaire
Les jeunes confrontés aux choix d’orientation ont besoin  d’une écoute attentive de la part de leurs parents. Les parents  ne sont pas forcément aux faits des filières de formation et  ne connaissent pas tous les métiers mais ils peuvent être au  jour le jour un soutien pour leur enfant qui s’interroge.
		\item \textit{\textbf{Les enseignants}} \\
		Dont l’action même d’enseigner est partie prenante de l’orientation et qui doivent aussi y exercer une mission précise.
Quoiqu’i en soit, en matière d’orientation, les enseignants exercent une influence déterminante par la seule mise œuvre de leurs démarches pédagogiques. Ils disposent, par les outils et les méthodes qu’ils utilisent ainsi que par leur comportement, des moyens d’agir sur certains éléments constitutifs de la maturation des choix de l’élève, particulièrement sur l’image qu’il se forge de lui. Les enseignants aident l’élève à prendre conscience  de ses potentialités et guide ses choix d’orientation.  
	\item \textit{\textbf{Le chef d’établissement}} \\
	Le chef d’établissement joue aussi un rôle fondamental dans la préparation de ses élèves à l’orientation. Il est responsable du dispositif d’orientation mis en place dans son établissement. C’est sous sa responsabilité que le programme d’information et d’orientation s’élabore et s’applique. Il est le garant du respect des réglementations en vigueur, mais au-delà il imprime selon ses convictions son expérience et les caractéristiques des populations dont il a la charge, le mouvement qui induit généralement à une véritable politique de l’information et de l’orientation. Il a un rôle particulièrement important dans la mise en œuvre des procédures d’orientations, dans le déroulement du dialogue avec les parents et dans le suivi des décisions.
	\item \textit{\textbf{Le conseiller d'orientation}} \\
	Les tâches du conseiller d’orientation dans une école s’avèrent multiples et difficiles à accomplir par lui seul car c’est un travail qui requiert la collaboration de différents acteurs comme nous l’avons signalé ci-dessus. L’une de ses missions consiste à fournit les informations.
	Le processus de cette action d’information est de :
	-	Mettre l’élève en état de réceptivité ;
	-	Proposer des informations en rapport avec les intérêts connus ;
	-	Adapter le message au niveau de l’élève et pour cela, il faut donner des propositions supplémentaires et faire des synthèses nécessaires avec la participation de l’élève.
	Le but à atteindre est de mettre l’élève en état de réaliser une self-orientation. De plus, le conseiller a le devoir de fournir aux parents des informations nécessaires pour qu’ils puissent assurer leurs responsabilités vis-à-vis de leurs enfants.
\end{enumerate}
\section{Etude des procédures d’orientation }
L’orientation et les formations proposées  aux élèves tiennent compte de leurs  aspirations, de leurs aptitudes et des  perspectives professionnelles liées aux besoins  prévisibles de la société, de l’économie et de  l’aménagement du territoire. \\
Dans ce cadre,  les élèves ou les étudiants élaborent leur projet d’orientation  scolaire et professionnelle avec l’aide des  parents, des enseignants, des personnels  d’orientation et des autres professionnels compétents, etc. Le chef d’établissement, décideur final, en l’état actuel du système, c’est lui qui prend la décision finale d’orientation, sur  proposition des membres du conseil de classe.
\begin{itemize}[label=\textbullet]
	\litem{Un conseil personnalisé : } \\
	Cette méthode s’appuie sur l’échange et le dialogue et permet de déceler des potentiels sans intervenir de manière directive dans les réponses et les choix du candidat.
 
           A l’aide de tests validés par des psychologues d’une part, et d’entretiens non directifs d’autre part, le conseiller définit des profils de métiers et d’études en adéquation avec les motivations, les aptitudes et la capacité de travail. Il tient compte des débouchés réels offerts par le marché du travail. Cette méthode permet de s’adapter et de respecter la personnalité de chacun en construisant ensemble un projet d’orientation.
	\litem{Le bilan d’orientation : } \\
	
	Destiné à tous les élèves à partir de la 3ème, ainsi qu'aux étudiants en recherche de réorientation, ce bilan permet de prendre les bonnes décisions d’orientation, en toute connaissance de cause. Le bilan d'orientation se déroule en 3 phases :
Phase 1 : Exploration
Ce premier rendez-vous permet de recueillir vos motivations et vos aspirations et d'évaluer vos aptitudes, etc. Cette phase s’appuie sur : des entretiens non directifs
 l'évaluation des aptitudes de l’étudiant ;  et un test concernant ses motivations, sa personnalité, ses  centres d'intérêt ...  \\
Phase 2 : Analyse et Synthèse
Le conseiller effectue une synthèse des résultats obtenus à l’aide d'un logiciel développé au Canada et utilisé depuis plus de 30 ans, etc. Ce logiciel a reçu le label \ac{RIP}  du ministère de l'Education Nationale, il est remis à jour chaque année. Il bâtit ensuite un projet d’orientation cohérent. \\
Phase 3 : Restitution des résultats
Lors du  deuxième rendez-vous, le conseiller propose à l’étudiant une sélection de métiers et de formations. Un rapport écrit contenant des fiches métiers (descriptif, niveau de formation à atteindre, perspectives d'emploi ...) et des informations sur les formations à suivre vous est remis.

\end{itemize}

\section{Etudes des documents utilisées}
\begin{itemize}[label=\textbullet]
	\item Dossiers scolaires et psychologiques \\
	Les données de toute étude pédagogique et psychologique en orientation  sont contenues dans le dossier scolaire et le dossier psychologique établis pour chaque élève à un moment de sa scolarité, puis complétés au cours de ses études à l’aide des informations scolaires, familiales, psychologiques, qui permettront au conseiller psychologue de suivre son évolution et de donner un conseil d’orientation au moment du choix scolaire ou professionnel.
Son but est de : 
	\begin{itemize}
		\item Connaitre l’élève au cours de son évolution ;
		\item Etudier les facteurs de réussite ou d’échec qui sont apparus au cours de cette évolution ;
		\item En rechercher les causes et trouver les remèdes s’il le faut
	Son contenu est  constitué par les données quantitatives et qualitatives.
		\begin{enumerate}
			\item Les données quantitatives : sont des résultats obtenus:\\
				\begin{itemize}
					\item D’une part aux examens écrits ou oraux.
					\item D’autre part aux tests d’aptitudes et de connaissances.
				\end{itemize}
			\item Les données qualitatives : \\
			 sont recueillies par les tests de personnalité et les entretiens. Elles sont complétées par les comptes rendus des conseils de classe, de délibérations….
		\end{enumerate}
	\end{itemize}
	\item Analyse du dossier psychologique\\
	Afin d’obtenir la vue la plus objective possible de l’enfant, les différents éléments des dossiers sont analysés de façon à permettre des confrontations et de recoupements
\begin{enumerate}
	\item les données physiologiques : \\
	vision, audition, motricité, fatigabilité, déficience, etc. sont obtenues grâce à l’étude :\\
	\begin{itemize}
		\item De la fiche médicale,
		\item Des questionnaires (élèves-parents-enseignants),
		\item Des examens psychologiques.
	\end{itemize}
	\item les données mentales :\\
	 efficience, mémoire (formes), intelligence (niveau, attitudes, facteurs spécifiques..,), aptitudes particulières (scientifiques, littéraires, artistiques…) sont apportées par :
	 	\begin{itemize}
	 	\item La fiche scolaire,
        \item Les questionnaires (élèves-parents),
		\item L’examen psychologique.
		\end{itemize}
	\item les données scolaires (niveau d’acquisition scolaire…) sont fournies par :\\
			\begin{itemize}
			\item La fiche scolaire ou bulletin,
			\item Les tests de connaissances scolaires.
			\end{itemize}
	\item le données caractérielles :\\
	activité ou passivité, sens de l’effort, émotivité, affectivité, attitude devant le travail, attitude à la maison, à l’école, attitude devant la réussite, devant l’échec, attitude envers les parents, enseignants, élèves…sont apportées par :
	\begin{itemize}
	\item	Les questionnaires,
	\item	Les entretiens,
   	\item    Les épreuves psychologiques particulières (les tests projectifs).
	\end{itemize}
\end{enumerate}
\end{itemize}
\paragraph{}
Notons également que quand le dialogue entre l'équipe éducative et la famille n’a pas permis d'aboutir à une  décision commune, les parents peuvent engager une procédure d'appel regroupant la commission d’appel. Cette dernière est présidée par  le directeur des services académiques, ayant l’objectif de réexaminer la décision d'orientation  en fonction des notes de l'élève, de ses capacités, de ses difficultés, de ses projets. Son point  de vue est extérieur.  Ainsi, les membres de la commission (chefs d'établissement, professeurs,  conseillers d’orientation...) ne sont pas directement impliqués dans l’histoire scolaire de  l’élève.
\section{L'orientation à l' \ac{ULPGL} /GOMA  \cite{OrBk6} }
Abordant la question de l’orientation à l’université libre des pays des grands lacs, l’orientation est assurée par l’apparitorat  central et la direction de scolarité. 
\paragraph{}
Dès son arrivé à l’université, on présente au candidat un papier sur lequel toutes les facultés sont mentionnées ainsi que les orientations, mais en plus les exigences pour chaque faculté. 
 A part cela, l’étudiant doit écrire une lettre manuscrite, et doit compléter une fiche de demande d’inscription, et déposer son dossier contenant : son diplôme d’Etat ou son équivalent, une attestation de naissance, une attestation de bonne vie et mœurs, une attestation de célibat, une attestation de nationalité, une attestation de résidence  (pour les étrangers) et un certificat d’aptitude physique. Après avoir fournis tous ces éléments, la direction de scolarité va siéger et analyser le dossier du candidat avant de le soumettre à un examen écrit.   
Un examen est organisé pour les étudiants ayant obtenu moins de soixante pourcent. Ce concours est organisé en deux moments. Le premier concours se déroule au mois de septembre et le deuxième concours avant la rentrée académique  pour permettre à ceux qui ont raté le premier examen de se rattraper et être éligible. Les résultats obtenus à ce concours sont affichés à la valve et la commission chargé des inscriptions, dirigé par le secrétaire général académique décide si le récipiendaire est retenu, rejette ou réorienté dans une autre faculté compte tenu de ses résultats obtenus. 
\paragraph{}
Concernant l’orientation proprement dite, la direction de scolarité explique aux étudiants comment fonctionne chaque faculté de l’université, les exigences, les débouchés sur le milieu professionnel et chaque étudiant choisi sa faculté compte tenu de ses choix, ses gouts, ses aspirations, ses motivations, ses objectifs et sa vision sans pour autant le contraindre. Cependant la contrainte intervient au cas où, l’étudiant a choisi une faculté qui ne correspond pas à sa capacité intellectuelle, mais en plus si l’étudiant échoue ou produit un résultat moins satisfaisant à la fin d’une année académique, la direction de scolarité décide de le mettre en  observation, après les examens du premier semestre ils peuvent décider de le réorienter dans une autre faculté où il peut s’en sortir mieux.  Si cette orientation ne produit pas des résultats satisfaisants la commission décide du refus, ou du renvoi de l’étudiant de l’université car il est  \ac{INAPS} ou il a épuisé toutes les possibilités qui lui ont étés offertes. 