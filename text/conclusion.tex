
\chapter*{Conclusion}
\addcontentsline{toc}{chapter}{Conclusion}
Tous au long de ce travail nous avons pu démontrer comment utiliser le data mining sur le données issues du système d'information \ac{UAT} pour aider les nouveaux étudiant à l'université à s'orienter.\\
Nous nous sommes posés 3 questions au départ : \\
- \emph{Comment utiliser les techniques  du data mining pour doter les universités des outils d'aide à la décision les permettant de bien orienter les étudiants dès leur entrée à l'université ?  }\\
- \emph{Comment peut-on aider les élèves finalistes du secondaires  à pouvoir faire le choix de leur filières à l'université ? }\\
- \emph{Les techniques du data mining peuvent - elles apporter leur contribution dans ce domaine ? }
\paragraph{}
Nous avons utilisé les diverses techniques du data mining ainsi que les statistiques inférentielle pour analyser  et nettoyer les données mise à notre disposition par les autorités de l'\ac{ULPGL} et nous avons pu découvrir qu'actuellement les étudiants se basent sur leur age , le sexe, l'école de provenance ainsi que l'option suivie à l'école secondaire pour s'orienter à l'université .
Sur base des résultats qu'un étudiant a obtenu lors de son cursus académique nous avons pu créer une variable appelé le \ac{CGPA}  qui  n'est rien d'autre que la moyenne des points obtenu durant son cursus. 
En effectuant les test statistiques nous avons constaté que cette variable ne dépend  que de 2 facteurs au sein de chaque faculté entre autre l'école de provenance , et l'option suive à l'école secondaire , et qu'il ne dépend ni du sexe, de l'age , et difficilement du pourcentage obtenu à l'\ac{EXETAT}.
\paragraph{}
Ainsi nous avons élaborer un modèle de prédiction   qui se base sur le 3 facteurs dont dépend le CGPA pour essayer de prédire celui - ci pour un nouveau étudiant à la faculté, pour s'y faire nous avons utilisé 5 techniques du Machine Learning dont 3 modèles de régression (Lasso, Ridge, ElasticNet) et 2 technique de \ac{SVR} (Lineaire et avec kernel RBF)  et à la fin nous avons combiné ces 5 modèle par la technique de stacking pour avoir une bonne prédiction.

Nous avons évalué le modèle en se basant sur l'erreur moyen quadratique \ac{RMSE} et une validation croisée et avons obtenu un score de moins de 10\% pour presque toutes les facultés.\\
Pour aider les finalistes à s'orienter nous avons mis à leur disposition une micro application web  qui prédit leur \ac{CGPA} au sein de chaque faculté et leur permet de choisir une faculté dans laquelle ils ont plus de chance de réussir .\\
Au final nous avons pu remarquer que les techniques du data mining on pu apporter leur contribution tant soi peux dans ce domaine. 
\paragraph{}
Quand au limites du travail soulignons qu'il n'est pas encore terminé , il nous manque certaines données pour automatiser toutes les taches et de débarrasser de l'expertiser humaine , entre autres diverses données liées au cursus secondaire comme soulignée au chapitre 2 ,  mais aussi d'une boucle de feedback pour permettre à notre modèle de s'entrainer avec des nouvelles variables et nouvelles données , c'est pourquoi le travail a été conçu selon la norme open-source pour permettre à tous chercheur d'y contribuer .