    
\documentclass[11pt]{article}
        
    \usepackage[T1]{fontenc}
    % Nicer default font (+ math font) than Computer Modern for most use cases
    \usepackage{mathpazo}

    % Basic figure setup, for now with no caption control since it's done
    % automatically by Pandoc (which extracts ![](path) syntax from Markdown).
    \usepackage{graphicx}
    % We will generate all images so they have a width \maxwidth. This means
    % that they will get their normal width if they fit onto the page, but
    % are scaled down if they would overflow the margins.
    \makeatletter
    \def\maxwidth{\ifdim\Gin@nat@width>\linewidth\linewidth
    \else\Gin@nat@width\fi}
    \makeatother
    \let\Oldincludegraphics\includegraphics
    % Set max figure width to be 80% of text width, for now hardcoded.
    \renewcommand{\includegraphics}[1]{\Oldincludegraphics[width=.8\maxwidth]{#1}}
    % Ensure that by default, figures have no caption (until we provide a
    % proper Figure object with a Caption API and a way to capture that
    % in the conversion process - todo).
    \usepackage{caption}
    \DeclareCaptionLabelFormat{nolabel}{}
    \captionsetup{labelformat=nolabel}

    \usepackage{adjustbox} % Used to constrain images to a maximum size 
    \usepackage{xcolor} % Allow colors to be defined
    \usepackage{enumerate} % Needed for markdown enumerations to work
    \usepackage{geometry} % Used to adjust the document margins
    \usepackage{amsmath} % Equations
    \usepackage{amssymb} % Equations
    \usepackage{textcomp} % defines textquotesingle
    % Hack from http://tex.stackexchange.com/a/47451/13684:
    \AtBeginDocument{%
        \def\PYZsq{\textquotesingle}% Upright quotes in Pygmentized code
    }
    \usepackage{upquote} % Upright quotes for verbatim code
    \usepackage{eurosym} % defines \euro
    \usepackage[mathletters]{ucs} % Extended unicode (utf-8) support
    \usepackage[utf8x]{inputenc} % Allow utf-8 characters in the tex document
    \usepackage{fancyvrb} % verbatim replacement that allows latex
    \usepackage{grffile} % extends the file name processing of package graphics 
                         % to support a larger range 
    % The hyperref package gives us a pdf with properly built
    % internal navigation ('pdf bookmarks' for the table of contents,
    % internal cross-reference links, web links for URLs, etc.)
    \usepackage{hyperref}
    \usepackage{longtable} % longtable support required by pandoc >1.10
    \usepackage{booktabs}  % table support for pandoc > 1.12.2
    \usepackage[inline]{enumitem} % IRkernel/repr support (it uses the enumerate* environment)
    \usepackage[normalem]{ulem} % ulem is needed to support strikethroughs (\sout)
                                % normalem makes italics be italics, not underlines
    % Colors for the hyperref package
    \definecolor{urlcolor}{rgb}{0,.145,.698}
    \definecolor{linkcolor}{rgb}{.71,0.21,0.01}
    \definecolor{citecolor}{rgb}{.12,.54,.11}

    % ANSI colors
    \definecolor{ansi-black}{HTML}{3E424D}
    \definecolor{ansi-black-intense}{HTML}{282C36}
    \definecolor{ansi-red}{HTML}{E75C58}
    \definecolor{ansi-red-intense}{HTML}{B22B31}
    \definecolor{ansi-green}{HTML}{00A250}
    \definecolor{ansi-green-intense}{HTML}{007427}
    \definecolor{ansi-yellow}{HTML}{DDB62B}
    \definecolor{ansi-yellow-intense}{HTML}{B27D12}
    \definecolor{ansi-blue}{HTML}{208FFB}
    \definecolor{ansi-blue-intense}{HTML}{0065CA}
    \definecolor{ansi-magenta}{HTML}{D160C4}
    \definecolor{ansi-magenta-intense}{HTML}{A03196}
    \definecolor{ansi-cyan}{HTML}{60C6C8}
    \definecolor{ansi-cyan-intense}{HTML}{258F8F}
    \definecolor{ansi-white}{HTML}{C5C1B4}
    \definecolor{ansi-white-intense}{HTML}{A1A6B2}

    % commands and environments needed by pandoc snippets
    % extracted from the output of `pandoc -s`
    \providecommand{\tightlist}{%
      \setlength{\itemsep}{0pt}\setlength{\parskip}{0pt}}
    \DefineVerbatimEnvironment{Highlighting}{Verbatim}{commandchars=\\\{\}}
    % Add ',fontsize=\small' for more characters per line
    \newenvironment{Shaded}{}{}
    \newcommand{\KeywordTok}[1]{\textcolor[rgb]{0.00,0.44,0.13}{\textbf{{#1}}}}
    \newcommand{\DataTypeTok}[1]{\textcolor[rgb]{0.56,0.13,0.00}{{#1}}}
    \newcommand{\DecValTok}[1]{\textcolor[rgb]{0.25,0.63,0.44}{{#1}}}
    \newcommand{\BaseNTok}[1]{\textcolor[rgb]{0.25,0.63,0.44}{{#1}}}
    \newcommand{\FloatTok}[1]{\textcolor[rgb]{0.25,0.63,0.44}{{#1}}}
    \newcommand{\CharTok}[1]{\textcolor[rgb]{0.25,0.44,0.63}{{#1}}}
    \newcommand{\StringTok}[1]{\textcolor[rgb]{0.25,0.44,0.63}{{#1}}}
    \newcommand{\CommentTok}[1]{\textcolor[rgb]{0.38,0.63,0.69}{\textit{{#1}}}}
    \newcommand{\OtherTok}[1]{\textcolor[rgb]{0.00,0.44,0.13}{{#1}}}
    \newcommand{\AlertTok}[1]{\textcolor[rgb]{1.00,0.00,0.00}{\textbf{{#1}}}}
    \newcommand{\FunctionTok}[1]{\textcolor[rgb]{0.02,0.16,0.49}{{#1}}}
    \newcommand{\RegionMarkerTok}[1]{{#1}}
    \newcommand{\ErrorTok}[1]{\textcolor[rgb]{1.00,0.00,0.00}{\textbf{{#1}}}}
    \newcommand{\NormalTok}[1]{{#1}}
    
    % Additional commands for more recent versions of Pandoc
    \newcommand{\ConstantTok}[1]{\textcolor[rgb]{0.53,0.00,0.00}{{#1}}}
    \newcommand{\SpecialCharTok}[1]{\textcolor[rgb]{0.25,0.44,0.63}{{#1}}}
    \newcommand{\VerbatimStringTok}[1]{\textcolor[rgb]{0.25,0.44,0.63}{{#1}}}
    \newcommand{\SpecialStringTok}[1]{\textcolor[rgb]{0.73,0.40,0.53}{{#1}}}
    \newcommand{\ImportTok}[1]{{#1}}
    \newcommand{\DocumentationTok}[1]{\textcolor[rgb]{0.73,0.13,0.13}{\textit{{#1}}}}
    \newcommand{\AnnotationTok}[1]{\textcolor[rgb]{0.38,0.63,0.69}{\textbf{\textit{{#1}}}}}
    \newcommand{\CommentVarTok}[1]{\textcolor[rgb]{0.38,0.63,0.69}{\textbf{\textit{{#1}}}}}
    \newcommand{\VariableTok}[1]{\textcolor[rgb]{0.10,0.09,0.49}{{#1}}}
    \newcommand{\ControlFlowTok}[1]{\textcolor[rgb]{0.00,0.44,0.13}{\textbf{{#1}}}}
    \newcommand{\OperatorTok}[1]{\textcolor[rgb]{0.40,0.40,0.40}{{#1}}}
    \newcommand{\BuiltInTok}[1]{{#1}}
    \newcommand{\ExtensionTok}[1]{{#1}}
    \newcommand{\PreprocessorTok}[1]{\textcolor[rgb]{0.74,0.48,0.00}{{#1}}}
    \newcommand{\AttributeTok}[1]{\textcolor[rgb]{0.49,0.56,0.16}{{#1}}}
    \newcommand{\InformationTok}[1]{\textcolor[rgb]{0.38,0.63,0.69}{\textbf{\textit{{#1}}}}}
    \newcommand{\WarningTok}[1]{\textcolor[rgb]{0.38,0.63,0.69}{\textbf{\textit{{#1}}}}}
    
    
    % Define a nice break command that doesn't care if a line doesn't already
    % exist.
    \def\br{\hspace*{\fill} \\* }
    % Math Jax compatability definitions
    \def\gt{>}
    \def\lt{<}
    % Document parameters
    \title{DataExplorationDraft1}
    
    
    

    % Pygments definitions
    
\makeatletter
\def\PY@reset{\let\PY@it=\relax \let\PY@bf=\relax%
    \let\PY@ul=\relax \let\PY@tc=\relax%
    \let\PY@bc=\relax \let\PY@ff=\relax}
\def\PY@tok#1{\csname PY@tok@#1\endcsname}
\def\PY@toks#1+{\ifx\relax#1\empty\else%
    \PY@tok{#1}\expandafter\PY@toks\fi}
\def\PY@do#1{\PY@bc{\PY@tc{\PY@ul{%
    \PY@it{\PY@bf{\PY@ff{#1}}}}}}}
\def\PY#1#2{\PY@reset\PY@toks#1+\relax+\PY@do{#2}}

\expandafter\def\csname PY@tok@gd\endcsname{\def\PY@tc##1{\textcolor[rgb]{0.63,0.00,0.00}{##1}}}
\expandafter\def\csname PY@tok@gu\endcsname{\let\PY@bf=\textbf\def\PY@tc##1{\textcolor[rgb]{0.50,0.00,0.50}{##1}}}
\expandafter\def\csname PY@tok@gt\endcsname{\def\PY@tc##1{\textcolor[rgb]{0.00,0.27,0.87}{##1}}}
\expandafter\def\csname PY@tok@gs\endcsname{\let\PY@bf=\textbf}
\expandafter\def\csname PY@tok@gr\endcsname{\def\PY@tc##1{\textcolor[rgb]{1.00,0.00,0.00}{##1}}}
\expandafter\def\csname PY@tok@cm\endcsname{\let\PY@it=\textit\def\PY@tc##1{\textcolor[rgb]{0.25,0.50,0.50}{##1}}}
\expandafter\def\csname PY@tok@vg\endcsname{\def\PY@tc##1{\textcolor[rgb]{0.10,0.09,0.49}{##1}}}
\expandafter\def\csname PY@tok@vi\endcsname{\def\PY@tc##1{\textcolor[rgb]{0.10,0.09,0.49}{##1}}}
\expandafter\def\csname PY@tok@vm\endcsname{\def\PY@tc##1{\textcolor[rgb]{0.10,0.09,0.49}{##1}}}
\expandafter\def\csname PY@tok@mh\endcsname{\def\PY@tc##1{\textcolor[rgb]{0.40,0.40,0.40}{##1}}}
\expandafter\def\csname PY@tok@cs\endcsname{\let\PY@it=\textit\def\PY@tc##1{\textcolor[rgb]{0.25,0.50,0.50}{##1}}}
\expandafter\def\csname PY@tok@ge\endcsname{\let\PY@it=\textit}
\expandafter\def\csname PY@tok@vc\endcsname{\def\PY@tc##1{\textcolor[rgb]{0.10,0.09,0.49}{##1}}}
\expandafter\def\csname PY@tok@il\endcsname{\def\PY@tc##1{\textcolor[rgb]{0.40,0.40,0.40}{##1}}}
\expandafter\def\csname PY@tok@go\endcsname{\def\PY@tc##1{\textcolor[rgb]{0.53,0.53,0.53}{##1}}}
\expandafter\def\csname PY@tok@cp\endcsname{\def\PY@tc##1{\textcolor[rgb]{0.74,0.48,0.00}{##1}}}
\expandafter\def\csname PY@tok@gi\endcsname{\def\PY@tc##1{\textcolor[rgb]{0.00,0.63,0.00}{##1}}}
\expandafter\def\csname PY@tok@gh\endcsname{\let\PY@bf=\textbf\def\PY@tc##1{\textcolor[rgb]{0.00,0.00,0.50}{##1}}}
\expandafter\def\csname PY@tok@ni\endcsname{\let\PY@bf=\textbf\def\PY@tc##1{\textcolor[rgb]{0.60,0.60,0.60}{##1}}}
\expandafter\def\csname PY@tok@nl\endcsname{\def\PY@tc##1{\textcolor[rgb]{0.63,0.63,0.00}{##1}}}
\expandafter\def\csname PY@tok@nn\endcsname{\let\PY@bf=\textbf\def\PY@tc##1{\textcolor[rgb]{0.00,0.00,1.00}{##1}}}
\expandafter\def\csname PY@tok@no\endcsname{\def\PY@tc##1{\textcolor[rgb]{0.53,0.00,0.00}{##1}}}
\expandafter\def\csname PY@tok@na\endcsname{\def\PY@tc##1{\textcolor[rgb]{0.49,0.56,0.16}{##1}}}
\expandafter\def\csname PY@tok@nb\endcsname{\def\PY@tc##1{\textcolor[rgb]{0.00,0.50,0.00}{##1}}}
\expandafter\def\csname PY@tok@nc\endcsname{\let\PY@bf=\textbf\def\PY@tc##1{\textcolor[rgb]{0.00,0.00,1.00}{##1}}}
\expandafter\def\csname PY@tok@nd\endcsname{\def\PY@tc##1{\textcolor[rgb]{0.67,0.13,1.00}{##1}}}
\expandafter\def\csname PY@tok@ne\endcsname{\let\PY@bf=\textbf\def\PY@tc##1{\textcolor[rgb]{0.82,0.25,0.23}{##1}}}
\expandafter\def\csname PY@tok@nf\endcsname{\def\PY@tc##1{\textcolor[rgb]{0.00,0.00,1.00}{##1}}}
\expandafter\def\csname PY@tok@si\endcsname{\let\PY@bf=\textbf\def\PY@tc##1{\textcolor[rgb]{0.73,0.40,0.53}{##1}}}
\expandafter\def\csname PY@tok@s2\endcsname{\def\PY@tc##1{\textcolor[rgb]{0.73,0.13,0.13}{##1}}}
\expandafter\def\csname PY@tok@nt\endcsname{\let\PY@bf=\textbf\def\PY@tc##1{\textcolor[rgb]{0.00,0.50,0.00}{##1}}}
\expandafter\def\csname PY@tok@nv\endcsname{\def\PY@tc##1{\textcolor[rgb]{0.10,0.09,0.49}{##1}}}
\expandafter\def\csname PY@tok@s1\endcsname{\def\PY@tc##1{\textcolor[rgb]{0.73,0.13,0.13}{##1}}}
\expandafter\def\csname PY@tok@dl\endcsname{\def\PY@tc##1{\textcolor[rgb]{0.73,0.13,0.13}{##1}}}
\expandafter\def\csname PY@tok@ch\endcsname{\let\PY@it=\textit\def\PY@tc##1{\textcolor[rgb]{0.25,0.50,0.50}{##1}}}
\expandafter\def\csname PY@tok@m\endcsname{\def\PY@tc##1{\textcolor[rgb]{0.40,0.40,0.40}{##1}}}
\expandafter\def\csname PY@tok@gp\endcsname{\let\PY@bf=\textbf\def\PY@tc##1{\textcolor[rgb]{0.00,0.00,0.50}{##1}}}
\expandafter\def\csname PY@tok@sh\endcsname{\def\PY@tc##1{\textcolor[rgb]{0.73,0.13,0.13}{##1}}}
\expandafter\def\csname PY@tok@ow\endcsname{\let\PY@bf=\textbf\def\PY@tc##1{\textcolor[rgb]{0.67,0.13,1.00}{##1}}}
\expandafter\def\csname PY@tok@sx\endcsname{\def\PY@tc##1{\textcolor[rgb]{0.00,0.50,0.00}{##1}}}
\expandafter\def\csname PY@tok@bp\endcsname{\def\PY@tc##1{\textcolor[rgb]{0.00,0.50,0.00}{##1}}}
\expandafter\def\csname PY@tok@c1\endcsname{\let\PY@it=\textit\def\PY@tc##1{\textcolor[rgb]{0.25,0.50,0.50}{##1}}}
\expandafter\def\csname PY@tok@fm\endcsname{\def\PY@tc##1{\textcolor[rgb]{0.00,0.00,1.00}{##1}}}
\expandafter\def\csname PY@tok@o\endcsname{\def\PY@tc##1{\textcolor[rgb]{0.40,0.40,0.40}{##1}}}
\expandafter\def\csname PY@tok@kc\endcsname{\let\PY@bf=\textbf\def\PY@tc##1{\textcolor[rgb]{0.00,0.50,0.00}{##1}}}
\expandafter\def\csname PY@tok@c\endcsname{\let\PY@it=\textit\def\PY@tc##1{\textcolor[rgb]{0.25,0.50,0.50}{##1}}}
\expandafter\def\csname PY@tok@mf\endcsname{\def\PY@tc##1{\textcolor[rgb]{0.40,0.40,0.40}{##1}}}
\expandafter\def\csname PY@tok@err\endcsname{\def\PY@bc##1{\setlength{\fboxsep}{0pt}\fcolorbox[rgb]{1.00,0.00,0.00}{1,1,1}{\strut ##1}}}
\expandafter\def\csname PY@tok@mb\endcsname{\def\PY@tc##1{\textcolor[rgb]{0.40,0.40,0.40}{##1}}}
\expandafter\def\csname PY@tok@ss\endcsname{\def\PY@tc##1{\textcolor[rgb]{0.10,0.09,0.49}{##1}}}
\expandafter\def\csname PY@tok@sr\endcsname{\def\PY@tc##1{\textcolor[rgb]{0.73,0.40,0.53}{##1}}}
\expandafter\def\csname PY@tok@mo\endcsname{\def\PY@tc##1{\textcolor[rgb]{0.40,0.40,0.40}{##1}}}
\expandafter\def\csname PY@tok@kd\endcsname{\let\PY@bf=\textbf\def\PY@tc##1{\textcolor[rgb]{0.00,0.50,0.00}{##1}}}
\expandafter\def\csname PY@tok@mi\endcsname{\def\PY@tc##1{\textcolor[rgb]{0.40,0.40,0.40}{##1}}}
\expandafter\def\csname PY@tok@kn\endcsname{\let\PY@bf=\textbf\def\PY@tc##1{\textcolor[rgb]{0.00,0.50,0.00}{##1}}}
\expandafter\def\csname PY@tok@cpf\endcsname{\let\PY@it=\textit\def\PY@tc##1{\textcolor[rgb]{0.25,0.50,0.50}{##1}}}
\expandafter\def\csname PY@tok@kr\endcsname{\let\PY@bf=\textbf\def\PY@tc##1{\textcolor[rgb]{0.00,0.50,0.00}{##1}}}
\expandafter\def\csname PY@tok@s\endcsname{\def\PY@tc##1{\textcolor[rgb]{0.73,0.13,0.13}{##1}}}
\expandafter\def\csname PY@tok@kp\endcsname{\def\PY@tc##1{\textcolor[rgb]{0.00,0.50,0.00}{##1}}}
\expandafter\def\csname PY@tok@w\endcsname{\def\PY@tc##1{\textcolor[rgb]{0.73,0.73,0.73}{##1}}}
\expandafter\def\csname PY@tok@kt\endcsname{\def\PY@tc##1{\textcolor[rgb]{0.69,0.00,0.25}{##1}}}
\expandafter\def\csname PY@tok@sc\endcsname{\def\PY@tc##1{\textcolor[rgb]{0.73,0.13,0.13}{##1}}}
\expandafter\def\csname PY@tok@sb\endcsname{\def\PY@tc##1{\textcolor[rgb]{0.73,0.13,0.13}{##1}}}
\expandafter\def\csname PY@tok@sa\endcsname{\def\PY@tc##1{\textcolor[rgb]{0.73,0.13,0.13}{##1}}}
\expandafter\def\csname PY@tok@k\endcsname{\let\PY@bf=\textbf\def\PY@tc##1{\textcolor[rgb]{0.00,0.50,0.00}{##1}}}
\expandafter\def\csname PY@tok@se\endcsname{\let\PY@bf=\textbf\def\PY@tc##1{\textcolor[rgb]{0.73,0.40,0.13}{##1}}}
\expandafter\def\csname PY@tok@sd\endcsname{\let\PY@it=\textit\def\PY@tc##1{\textcolor[rgb]{0.73,0.13,0.13}{##1}}}

\def\PYZbs{\char`\\}
\def\PYZus{\char`\_}
\def\PYZob{\char`\{}
\def\PYZcb{\char`\}}
\def\PYZca{\char`\^}
\def\PYZam{\char`\&}
\def\PYZlt{\char`\<}
\def\PYZgt{\char`\>}
\def\PYZsh{\char`\#}
\def\PYZpc{\char`\%}
\def\PYZdl{\char`\$}
\def\PYZhy{\char`\-}
\def\PYZsq{\char`\'}
\def\PYZdq{\char`\"}
\def\PYZti{\char`\~}
% for compatibility with earlier versions
\def\PYZat{@}
\def\PYZlb{[}
\def\PYZrb{]}
\makeatother


    % Exact colors from NB
    \definecolor{incolor}{rgb}{0.0, 0.0, 0.5}
    \definecolor{outcolor}{rgb}{0.545, 0.0, 0.0}



    
    % Prevent overflowing lines due to hard-to-break entities
    \sloppy 
    % Setup hyperref package
    \hypersetup{
      breaklinks=true,  % so long urls are correctly broken across lines
      colorlinks=true,
      urlcolor=urlcolor,
      linkcolor=linkcolor,
      citecolor=citecolor,
      }
    % Slightly bigger margins than the latex defaults
    
    \geometry{verbose,tmargin=1in,bmargin=1in,lmargin=1in,rmargin=1in}
    
    

    \begin{document}
    
    
    \maketitle
    
    
    \section{Presentation et Exploitation des
Données}\label{presentation-et-exploitation-des-donnuxe9es}

    Dans ce chapitre nous allons exploités les données mise en notre
disposition par les autorités de l'ULPGL. Celles-ci sont issues du
système d'information UAT (Univeristy Administrative Tool) et pour des
raisons de confidentialité nous n'avons pas eu accès a toute la base des
données nous avons juste fais une requete des donnes dont nous avons
besoin pour notre étude et l'administrateur a executé une requete vers
sa base des données et nous a fourni les données dont nous avions besoin
pour l'etude sous forme d'un fichier csv (commat separted values). Comme
souligné dans le chapitre premier ce chapitre se basera sur la
methodologie CRISP-DM elle sera subdivisé en differentes sections: -
L'exploration et la preparation des donnéés - selection des algorithmes
et leur execution - l'amelioratrion et optimisation des algorithmes
source : Sklean Handbook Appendix 2

    \subsection{Exploration et la preparation des donness
données}\label{exploration-et-la-preparation-des-donness-donnuxe9es}

    source :
https://www.analyticsvidhya.com/blog/2016/01/guide-data-exploration/
SUNIL RAY , JANUARY 10, 2016

    les specialites affirment que 70-80 \% du temps consacré à un projet
dataMining est alloué à la phase de l'exploration et la preparation des
données , il n' ya pas des racourcis pour cette phase et si on l'a pas
bien effectué nous risquons de nous retouver entrain d'ameliorer
l'exactitude de notre algorithme mais en vain nous serons toujours
obligées de retourner à cette phase et toutes ces techniques de
l'exploration des données pourrons nous venir en aide .

    \begin{enumerate}
\def\labelenumi{\arabic{enumi}.}
\tightlist
\item
  Les Etapes de la phase d'exploration ét la preparation des donnes
  Certaines des ces étapes sont mentionées sur la figure suivante ;
  source: http://www.saedsayad.com/data\_mining\_map.htm
\end{enumerate}

En bref l'exploration des données consiste à se plonger dans le passée
pour prédire l'avenir .Souvenenons nous que la qualité de notre entré
determine la qualité de notre sortie, ces phases nous permettent
d'ameliorer la qualité de notre entré en vue d'avoir une bonne sortie.
Voici les étapes de cette phases:

\begin{itemize}
\tightlist
\item
  Identification des variables
\item
  Statistique Descriptive
\item
  Analyse Bi-varié
\item
  Traitement des valeur maquantes
\item
  Traitement des deviations ou outliers
\item
  Transformation des variables
\item
  creation des nouvelles variables
\end{itemize}

Comme nous l'avons soulignées dans le chapitre 1 ce processus est un
processus iteratif et incremetale nous executerons cette phase 2 a 5
fois ou plus en vue d'avoir un bon modele

    \subsection{Identification des Données et des
Variables}\label{identification-des-donnuxe9es-et-des-variables}

    Comme soulignées dans la phase d'introduction les données mise à notre
disposition sont sous format csv et nous alons utilisé la librarie
pandas de python pour faire l'analyse , nous utiliserons aussi d'autres
libraies qui nous permetrons de faire les statistiques ainsi ques les
visualisations :

voici le code pour charger le librairies.

    Nous venons de charger les donnes et nous pouvons remarquer à quoi ils
ressemblent

            \begin{Verbatim}[commandchars=\\\{\}]
{\color{outcolor}Out[{\color{outcolor}52}]:} (9606, 18)
\end{Verbatim}
        
    Nous remarquons que les données sont stocké dans un e structure de type
matricielle appelé dataframe. source :
https://pandas.pydata.org/pandas-docs/stable/dsintro.html\#dataframe

un DataFrame selon la documentation officielle de pandas est une
structure des données bidementionnel avec des colonnes des données des
differentes types . Il peut etre compareé à une feuille de calcul excel
ou une table dans SQL

    la commande dataset.shape nous a permis de dire que notre emsemble
d'apprentissage de depart comprend 9606 lignes et 22 colones ! Analysons
de plus pret les colones

            \begin{Verbatim}[commandchars=\\\{\}]
{\color{outcolor}Out[{\color{outcolor}6}]:} Index([u'IDENTIFICATION', u'BIRTHDAY', u'NAME', u'DIPLOMDATE', u'DIPLOMTYPE',
               u'DIPLOMMENTION', u'DIPLOMPERCENTAGE', u'DIPLOMSECTION',
               u'DIPLOMOPTION', u'DIPLOMPLACE', u'SCHOOL', u'SCHOOLPROVINCE',
               u'SCHOOLCODE', u'SCHOOLSTATUS', u'ACADYEAR', u'PERC1', u'MENT1',
               u'PERC2', u'MENT2', u'FAC', u'OPT', u'PROM'],
              dtype='object')
\end{Verbatim}
        
    chaque ligne comprend les information d'un étudiant pour une année
Academique

    1 IDENTIFICATION : contient une identification unique et anonyme d'un
étudiant les noms et les matricules reeles des étudiants on éé cachées
pour des raisons de confidentialites

2 BIRTHDAY : contient la date de naissance de chaque étudiant

3 NAME : contient le sexe de chaque étudiant

4 DIPLOMDATE : L'anné d'optiention du diplome

5 DIPLOMTYPE : le type de diplome

6 DIPLOMMENTION : mention de diplome

7 DIPLOMPERCENTAGE: le pourcentage du diplome

8 DIPLOMSECTION: la section du diplome

9 DIPLOMOPTION : l'option

10 DIPLOMPLACE : l'endroit d'optention du diplome

11 SCHOOL : l'ecole de provenance

12 SCHOOLPROVINCE : la province de provenance

13 SCHOOLCODE : code de l'ecole

14 SCHOOLSTATUS : le status de l'ecole (privaté , publique ,
conventioneé ,..)

15 ACADYEAR : l'annéé academique '

16 PERC1 : poucentage en premiere session

17 MENT1 : mention en premier session

18 PERC2 : pourcentage en seconde session

19 MENT2 : mention en seconde session

20 FAC : la faculté

21 OPT : l'option

22 PROM : la promotion

    Comme nous pouvons le constater les colonnes 1-14 regrogent les
informations que chaque étudiant donné à son instcription , ils
constituerons nos variables d'entres les restes seronts utilisées pour
constituer notre variable de sortie

    Nous l'avons aussi signales que chaque ligne comprend les information
d'un étudiant pour une année academique . Pour mener bien notre anyse
nous alons grouper les information de chaque etudieant en une lignee

    nos données seront groupé selons les variables d'entrees ensuites les
donnés de sorties serot groupes selon une fonction d'aggregation
predefinie

    Nous allons premierement faire une analyse univarié sur les données en
entrees !

    Avant la phase de traitement des donnés regardons notre ensemble pour
enlever les colonnes sans informations consoiderables.

    Nous avons ecrit une function qui nous donnes un pourcentrage des
valeurs maquantes pour chaque collones et de prime à bord nous allons
supprimer certaines colonnes qu ne comportent pas des informations.

            \begin{Verbatim}[commandchars=\\\{\}]
{\color{outcolor}Out[{\color{outcolor}6}]:} BIRTHDAY             0
        DIPLOMDATE          98
        DIPLOMPLACE         98
        SCHOOLSTATUS         0
        DIPLOMPERCENTAGE     0
        SCHOOLPROVINCE       0
        FAC                  0
        ACADYEAR             0
        DIPLOMMENTION       99
        SCHOOLCODE          70
        PERC1               55
        PERC2               36
        DIPLOMOPTION         0
        OPT                  0
        SCHOOL               5
        NAME                 0
        DIPLOMSECTION        0
        MENT1                0
        PROM                 0
        MENT2               24
        IDENTIFICATION       0
        DIPLOMTYPE           0
        Name: \% of missing, dtype: int64
\end{Verbatim}
        
    Avec cette table nous remarquons que 4 colones ne nous servirons à riens
dans la suite car elles disposent de plus de 70 \% des valeurs maquantes
et nous devons les suprimées avant de continuer notre analyse

    Nous pouvons remarquer que notre ensemble d'apprentissage change de
dimenssion et (7216,22) à (7216,18)

            \begin{Verbatim}[commandchars=\\\{\}]
{\color{outcolor}Out[{\color{outcolor}16}]:} (9606, 18)
\end{Verbatim}
        
            \begin{Verbatim}[commandchars=\\\{\}]
{\color{outcolor}Out[{\color{outcolor}13}]:} Index([u'IDENTIFICATION', u'BIRTHDAY', u'NAME', u'DIPLOMTYPE',
                u'DIPLOMPERCENTAGE', u'DIPLOMSECTION', u'DIPLOMOPTION', u'SCHOOL',
                u'SCHOOLPROVINCE', u'SCHOOLSTATUS', u'ACADYEAR', u'PERC1', u'MENT1',
                u'PERC2', u'MENT2', u'FAC', u'OPT', u'PROM'],
               dtype='object')
\end{Verbatim}
        
    les colonnes en entrée deviennent les 10 premiers colonnes

    Pour une premiere approche nous allons grouper notre ensemble en
fonction des données en entré et ensuite ecrire une functuion qui va
groupper les donnes de sortie

            \begin{Verbatim}[commandchars=\\\{\}]
{\color{outcolor}Out[{\color{outcolor}35}]:} (9606, 18)
\end{Verbatim}
        
    voici la function qui va grouper les données de sortie

            \begin{Verbatim}[commandchars=\\\{\}]
{\color{outcolor}Out[{\color{outcolor}37}]:} (4715, 8)
\end{Verbatim}
        
    apres groupement en fonction des matricules nous venons de remarquer que
notre ensemble comprend 4715 rows et 18 columns et c'est sera notre
ensemble pour notre étude

    cette ensemble est subdivisé en variables d'entré et variables de
sortie!

            \begin{Verbatim}[commandchars=\\\{\}]
{\color{outcolor}Out[{\color{outcolor}24}]:} (4715, 10)
\end{Verbatim}
        
    Nous allons afficher notre ensemble d"enntre et de sortie ici

    Nous venons de finir avec la presentation des nos données nous allons
maintenant debuter avec la phase d'analyse promremendite des donnés que
nous avons en entré et ensuite nous fairons une analyse des données en
sortie en enfin analyse les donnes des sortie combinées à celles des
données ?

    \subsubsection{Analyse des données}\label{analyse-des-donnuxe9es}

    Cette phase comprend une analyse statistique bivarié et univarié nous
visualiserons les résultat à l'aide des graphiques . Dans cette partie
nous utiliserons beaucoup plus la statistique descriptives et
inferentielle.

    Comme nous pous pouvons le remarquer notre ensemble d'apprentissage
comprend à la fois des données numeriques (continues ) ainsi que des
données discrètes categories. voici comment nous allons procèder

    \paragraph{Statistique Descriptive}\label{statistique-descriptive}

    \begin{enumerate}
\def\labelenumi{\arabic{enumi}.}
\tightlist
\item
  variable Numériques ou continues : Pour les données continues nous
  allons essayer de comprendre la tendence et la dispertion des nos
  variables .les metriques utilisées sont sur la figure suivante: En
  bréf nous allons éxaminer le moyenne , le mode , l'ecart-type et la
  variance , nous conterons aussi les variables nous fairons les
  visualisations avec des boxplot! cette étape nous sera aussi utile
  dans le traitement des valeur maquantes et des outliers!
\item
  variable categorielle ou quantitative Pour les données discrètes nous
  aloons les tables des frequences pour comprendre la distrubution de
  chaque categorie nous pour aussi voir le pourcentage de chaque
  categorie , les histogrammes et bar chart seront utilisées.
\end{enumerate}

    \paragraph{Analyse Bivariée}\label{analyse-bivariuxe9e}

    c'est une technique d'analyse statistique des données, consistant à
découvrir la relation pouvant exister entre 2 variables dans le but de
tester l'hypothèse d'association et de causalité entre 2 variables !Par
exemple dans notre anlyser nous allons essayer de voir la relation
existant entre le choix de la faculté de le pourcentage du diplome à
l'exetat. Elle se deroule en 4 étapes : - Définition de la nature des
rélations - Identification et direction des rélations - Determination si
la relation est important du poinr de vue statistique(Intervalle de
confiance) - Detarmination de La force de relation

Source: Becker, William. Uncertainty propagation through large nonlinear
models. Diss. University of Sheffield, 2011. de Smith, M. J. "STATSREF:
Statistical Analysis Handbook-a web-based statistics." (2015).

    Nous effecurons cette analyse à 3 niveau : 1. Variables Continues et
categorielle ou quantitatives Pour effectuer cette analyse nous
utiliserons le test ANOVA (Analyse of variance):


{[}formule et decision{]} 2. variable Categorielles et Cateorielles

Pour ces types des données nous allons effectué le test de chi carré: Le
chi carré est un test statistique conçu pour déterminer si la différence
entre deux distributions de fréquences est attribuable à l'erreur
d'échantillonnage (le hasard) ou est suffisamment grande pour être
statistiquement significative.

Ho - est, comme son nom l'indique, une hypothèse qui postule qu'il n'y a
pas de différence entre les fréquences ou les proportions des deux
groupes elle est considére comme hypothèse nulle.

Si la différence entre les deux distributions est réduite, l'hypothèse
nulle sera acceptée. Si la différence est grande, l'hypothèse nulle sera
rejetée. Dans ce dernier cas, on parlera d'une différence
statistiquement significative parce que l'écart entre les deux
distributions est trop important pour être expliqué par le hasard
seulement : une différence réelle existe donc. {[}Inserrer la formule{]}

\begin{enumerate}
\def\labelenumi{\arabic{enumi}.}
\setcounter{enumi}{2}
\tightlist
\item
  Variables Continues et Continues
\end{enumerate}

Pour les variables continues on utilise cherche la correlation et pour
notre travail nous allons utilisée le coeficient de correlation de
pearson: Les coefficients de corrélation permettent de donner une mesure
synthétique de l'intensité de la relation entre deux caractères et de
son sens lorsque cette relation est monotone. Le coefficient de
corrélation de Pearson permet d'analyser les relations linéaires et le
coefficient de corrélation de Spearman les relations non-linéaires
monotones. Il existe d'autres coefficients pour les relations
non-linéaires et non-monotones.

Signalons que python dispose des mutiples librairies pour effectuer ces
genres d'analyse.

    Commencons par l'analyse des données univariés sur les variables d'entré

    \begin{Verbatim}[commandchars=\\\{\}]
<class 'pandas.core.frame.DataFrame'>
RangeIndex: 4038 entries, 0 to 4037
Data columns (total 10 columns):
IDENTIFICATION      4038 non-null int64
BIRTHDAY            4038 non-null object
NAME                4038 non-null object
DIPLOMTYPE          4038 non-null object
DIPLOMPERCENTAGE    4038 non-null float64
DIPLOMSECTION       4038 non-null object
DIPLOMOPTION        4038 non-null object
SCHOOL              4038 non-null object
SCHOOLPROVINCE      4038 non-null object
SCHOOLSTATUS        4038 non-null object
dtypes: float64(1), int64(1), object(8)
memory usage: 315.5+ KB

    \end{Verbatim}

    Nous remarquons que nous données en entré dispose des 10 colones avec
variables quantitatives et qualitatives,

    1. Variables continues

a. Attribue Date

    Type: String

    Pour nous faciliter la tache nous allons remplacer la date de naissance
de chaque individu par son age a ce moment ci et ainsi obtenit un
attribue continue de plus.

    Valeurs Maquantes : Oui , ils sont causées par des erreurs à l'entré

            \begin{Verbatim}[commandchars=\\\{\}]
{\color{outcolor}Out[{\color{outcolor}24}]:} 7
\end{Verbatim}
        
            \begin{Verbatim}[commandchars=\\\{\}]
{\color{outcolor}Out[{\color{outcolor}73}]:} 4038
\end{Verbatim}
        
            \begin{Verbatim}[commandchars=\\\{\}]
{\color{outcolor}Out[{\color{outcolor}76}]:} 0
\end{Verbatim}
        
    Solution : Ils sont remplace par la moyenne comme leur proportion est
insignifiant

    Ce tableau decrit toutes les informations possibles sur les données
continues et de prime à bord nous sommes à mesure de constater certaines
incoherences sur les diplome percentage qui on un maximun de 6053 et un
minimum de 0 qui est vraiment impossible car le diplome en RDC doi etre
compris entre 50 et 100 \% !Nous allons visualisé ces inchoherence de
plus prêt avec des boxplots.

    \begin{center}
    \adjustimage{max size={0.9\linewidth}{0.9\paperheight}}{DataExplorationDraft1_files/DataExplorationDraft1_73_0.png}
    \end{center}
    { \hspace*{\fill} \\}
    
    \begin{center}
    \adjustimage{max size={0.9\linewidth}{0.9\paperheight}}{DataExplorationDraft1_files/DataExplorationDraft1_74_0.png}
    \end{center}
    { \hspace*{\fill} \\}
    
    Au vu de ces courbes nous remarquons que l'attribue diplome percentage
dispose de beaucoup des deviations.

    Mais l'attribue Bithday a une distribution presque normale

            \begin{Verbatim}[commandchars=\\\{\}]
{\color{outcolor}Out[{\color{outcolor}36}]:} <matplotlib.axes.\_subplots.AxesSubplot at 0x7fb771ca6410>
\end{Verbatim}
        
    \begin{center}
    \adjustimage{max size={0.9\linewidth}{0.9\paperheight}}{DataExplorationDraft1_files/DataExplorationDraft1_77_1.png}
    \end{center}
    { \hspace*{\fill} \\}
    
    Nous pouvons facilement voir que l'age a une distribution presque
normale. NB: change norm hist to 1 to see count

    regardons de plus pret celui du diplome percentage

            \begin{Verbatim}[commandchars=\\\{\}]
{\color{outcolor}Out[{\color{outcolor}128}]:} <matplotlib.axes.\_subplots.AxesSubplot at 0x7f1a8e854150>
\end{Verbatim}
        
    \begin{center}
    \adjustimage{max size={0.9\linewidth}{0.9\paperheight}}{DataExplorationDraft1_files/DataExplorationDraft1_80_1.png}
    \end{center}
    { \hspace*{\fill} \\}
    
    A cause des dispertions nous ne pouvons pas bien visualis la
distribution. essayons d'isoler les distibution pour voir de plus prét
les données .

            \begin{Verbatim}[commandchars=\\\{\}]
{\color{outcolor}Out[{\color{outcolor}98}]:} 34
\end{Verbatim}
        
    Nous remarquons que nous avons 29 échantillons avec des valeur hors
normes nous allons les normaliser dans la phase de préparation

    \subparagraph{Variables qualitatives ou
categorieles}\label{variables-qualitatives-ou-categorieles}

    Pour chaque attriblue nous alons faire de count plot voir les
diffrerents valeurs

            \begin{Verbatim}[commandchars=\\\{\}]
{\color{outcolor}Out[{\color{outcolor}37}]:} ['BIRTHDAY',
          'NAME',
          'DIPLOMTYPE',
          'DIPLOMSECTION',
          'DIPLOMOPTION',
          'SCHOOL',
          'SCHOOLPROVINCE',
          'SCHOOLSTATUS']
\end{Verbatim}
        
            \begin{Verbatim}[commandchars=\\\{\}]
{\color{outcolor}Out[{\color{outcolor}39}]:} H    2771
         F    1944
         Name: NAME, dtype: int64
\end{Verbatim}
        
    Nous pouvons remarqué facilement avec cete comande la repartition des
sexes!

    \begin{center}
    \adjustimage{max size={0.9\linewidth}{0.9\paperheight}}{DataExplorationDraft1_files/DataExplorationDraft1_91_0.png}
    \end{center}
    { \hspace*{\fill} \\}
    
    Voici la repartition de sexes dans notre ensemble d'apprentissage on
peut aisement constater qu'il n'est pas si desequilibré que ça!, le
genre est vraiment respecté avec 41\% dés nouveaux étudiant etant de
sexe feminin.

    Attribue : Diplome Type

    \begin{center}
    \adjustimage{max size={0.9\linewidth}{0.9\paperheight}}{DataExplorationDraft1_files/DataExplorationDraft1_94_0.png}
    \end{center}
    { \hspace*{\fill} \\}
    
    Nous pouvons remarquer que notre dataset contient plus de 70\% d'element
avec le diplome d'etat et 26 avec un diplome type inconue nous allons
traaiter cela à la suite et quelque echantillon avec des diplomes du
Rwanda et d'autre avec des anciens diplomes, les individus dont le
diplomé type sont inconu serons consideré comme diplome d'etat, donc
nous pouvons le remplacer par le diplome d'etat.

            \begin{Verbatim}[commandchars=\\\{\}]
{\color{outcolor}Out[{\color{outcolor}50}]:} DIPLÔME D'ETAT              4706
         DIPLÔME D'ETAT A2              4
         DIPLÔME D'ETAT DU RWANDA       4
         DIPLÔME D'ETAT067              1
         Name: DIPLOMTYPE, dtype: int64
\end{Verbatim}
        
    \begin{center}
    \adjustimage{max size={0.9\linewidth}{0.9\paperheight}}{DataExplorationDraft1_files/DataExplorationDraft1_98_0.png}
    \end{center}
    { \hspace*{\fill} \\}
    
    Attribue Diplome option et Diplome Section

            \begin{Verbatim}[commandchars=\\\{\}]
{\color{outcolor}Out[{\color{outcolor}104}]:} 90
\end{Verbatim}
        
    Nous avons plus de 90 sections

    une chose importante à remarquer au niveau des attribues 'DIPLOMOPTION'
et 'DIPLOMSECTION' sont tres desorganisées il faut bien les oragniées
dans la phase de préparation des données

    Attribue School Province

    \begin{center}
    \adjustimage{max size={0.9\linewidth}{0.9\paperheight}}{DataExplorationDraft1_files/DataExplorationDraft1_104_0.png}
    \end{center}
    { \hspace*{\fill} \\}
    
    Avec cette image nous pouvons aisement constater que 63 \% des étudaints
de l'ulpgl proviennent de la province du nord kivu mais il ya une autre
categorie qui provient du sud KIVu soit 15\%

    Attribue SCHOOL ET SCHOOL STATUS

    Pour une bonne visualisation on peut prealablement transformer les
colones en minuscule

    Nous avon un attribue avec 1285 categories differentes ... Nous devons
les analyser en details

            \begin{Verbatim}[commandchars=\\\{\}]
{\color{outcolor}Out[{\color{outcolor}46}]:} inconnu         3833
         catholique       289
         protestant       260
         privé            170
         publique         152
         musulman           6
         kimbanguiste       3
         autodidacte        2
         Name: SCHOOLSTATUS, dtype: int64
\end{Verbatim}
        
    ces attribues aussi necessite un nettoyage et une bonne reparation

    Nous allons creer une branches à part pour le traitement de ces valeurs
pour les diplomes section et diplome option

            \begin{Verbatim}[commandchars=\\\{\}]
{\color{outcolor}Out[{\color{outcolor}47}]:} ['IDENTIFICATION',
          'BIRTHDAY',
          'NAME',
          'DIPLOMTYPE',
          'DIPLOMPERCENTAGE',
          'DIPLOMSECTION',
          'DIPLOMOPTION',
          'SCHOOL',
          'SCHOOLPROVINCE',
          'SCHOOLSTATUS']
\end{Verbatim}
        
    les collones que nous allons traiter sont school ,schoolstatus,
dilpomesection,diplomeoption

            \begin{Verbatim}[commandchars=\\\{\}]
{\color{outcolor}Out[{\color{outcolor}78}]:}    IDENTIFICATION         DIPLOMSECTION                   DIPLOMOPTION  \textbackslash{}
         0              45          PEDAGOGIEQUE                  PEDA GENERALE   
         1             215          SCIENTIFIQUE                  MATH-PHYSIQUE   
         2             343          SCIENTIFIQUE                  MATH PHYSIQUE   
         3             356  ECONOMIE ET COMMERCE                       ECONOMIE   
         4             429             TECHNIQUE  COMMERCIALE ET ADMINISTRATIVE   
         
                        SCHOOL SCHOOLSTATUS  
         0  INSTITUT MAENDELEO      inconnu  
         1      INSTITUT VUNGI      inconnu  
         2     INSTITUT FARAJA      inconnu  
         3       ESISE/GISENYI      inconnu  
         4          C,S, UMOJA      inconnu  
\end{Verbatim}
        
    Nous allons sauvergarder cet nouveau dataframe dans un fichier csv et
creer une notebook à part pour le traitement et le raffinement de ces
collones

    \subparagraph{Nettoyage des attribues
bruitées}\label{nettoyage-des-attribues-bruituxe9es}

    Nous avons pu remarquer dans la phase de precedante que certaines
attribues ont des valeurs très desorganisées et vraiment dispersé et ce
qui a une mauvaise influance sur le calcul de l'entropie et ainsi sur
les algorithmes du Machine Learning . Nous pouvons aisement constater
que ce problèmes est du à des fautes d'orthographes commise lors de la
phase de saisie des données et ainsi pour countinues nous devons essayer
de corriger ces erreurs et bien organisé les données . Voyons d'abord en
chiffre comment cela se presente

    Attribue Diplomes sections

            \begin{Verbatim}[commandchars=\\\{\}]
{\color{outcolor}Out[{\color{outcolor}42}]:} 101
\end{Verbatim}
        
    Avec cette commande nous remarquons que cette attribue dispose de 101
valeurs disctinctes qui c'es qui es impossible pour la valeur de diplome
section

    voyons un peu en details ce qui contient cette attribue

    Dans cette simple description nous remarquons que les valeurs on été mal
saisi comme par example les valeurs suivantes : 'TECSC', 'Techniqe',
'TECHN IQUE',technique','TCH' qui sont saisie pour la meme et unique
section 'techniques' mais avec differentes erreurs d'orthographe

    cela n'est qu'un example des differentes valeurs mal orthographiées
presente dans notre ensemble d'etude

    ce genre d'erreur de notation à pour consequence le fait qu'il font
augmenter l'entropie de nos colonnes et ainsi penalisent nos algoritmes
surtout lorsqu'on travaille avec les arbres de décisions nous avons
procéde à un nettoyage automatique qui a consisté en un groupement des
valleurs proches en utilisant la distance de leveinstein :(source :
leveinstein) et le cllustering par l'agorithme d'affinity propagation,et
ainsi qu'un nettoyage manuelle pour arranger les données à la fin de
cette phase nous avons obtenus des données moyenement propres et bien
netoyer avec un entropie faibe. Nous pouvons le remarquer dans
l'ensemble d'apprentissager suivant que ces données sont bien grouper.

            \begin{Verbatim}[commandchars=\\\{\}]
{\color{outcolor}Out[{\color{outcolor}36}]:} protestant      1370
         catholique      1305
         publique         726
         inconnu          691
         privé            536
         autodidacte       44
         musulman          38
         kimbanguiste       5
         Name: SCHOOLSTATUS, dtype: int64
\end{Verbatim}
        
            \begin{Verbatim}[commandchars=\\\{\}]
{\color{outcolor}Out[{\color{outcolor}7}]:} protestant      1370
        catholique      1305
        publique         726
        inconnu          691
        privé            536
        autodidacte       44
        musulman          38
        kimbanguiste       5
        Name: SCHOOLSTATUS, dtype: int64
\end{Verbatim}
        
    Nous pouvons maintenant combinner cette ensemble avec notre ensemble
d'apprentissage de départ et ainsi continuer nore anlyse univarié poour
les collones avce des varaibles qualitatives

            \begin{Verbatim}[commandchars=\\\{\}]
{\color{outcolor}Out[{\color{outcolor}162}]:} protestant      1370
          catholique      1305
          publique         726
          inconnu          691
          privé            536
          autodidacte       44
          musulman          38
          kimbanguiste       5
          Name: SCHOOLSTATUS, dtype: int64
\end{Verbatim}
        
    Nous allons enfin continuer avec notre anlyser univarié pour les
attribues nouvellement nettoyer

            \begin{Verbatim}[commandchars=\\\{\}]
{\color{outcolor}Out[{\color{outcolor}12}]:}    IDENTIFICATION DIPLOMSECTION DIPLOMOPTION    SCHOOL SCHOOLSTATUS  \textbackslash{}
         0          3895.0         TECHN           ca  i zanner   protestant   
         
           SCHOOL\_CORRECT SCHOOL\_RIGHT         OPTION\_RIGHT  
         0         zanner       zanner  commmerciale et adm  
\end{Verbatim}
        
    \subparagraph{Attribue SCHOOLSTATUS}\label{attribue-schoolstatus}

    Dans cette figure nous pouvons remarquer aisement que 29\% des étudiants
proviennent des écoles dites protestantes , 27\% viennent des écoles
catholiques , 11\% des écoles privé , 15 des écoles publiques mais aussi
il ya des étudiants venant des autodidactes , ceux provenant des écoles
musulmanes et kibanguistes mais en proportion vraiment négligeable

    \subparagraph{Attribue OPTIONRIGHT}\label{attribue-optionright}

    Nous allons maintenant nous attaquer à l'attribue option du diplome qui
contient les valeurs de l'option du diplome de 'étudaint'

    Essayons de voir de plus pret combien des valeur distinctes il dispose

            \begin{Verbatim}[commandchars=\\\{\}]
{\color{outcolor}Out[{\color{outcolor}32}]:} 33
\end{Verbatim}
        
    Nous pouvons aisement remarquer que les étudiants de notre étude
proviennent des 33 écoles differentes

    \begin{center}
    \adjustimage{max size={0.9\linewidth}{0.9\paperheight}}{DataExplorationDraft1_files/DataExplorationDraft1_149_0.png}
    \end{center}
    { \hspace*{\fill} \\}
    
    Dans cette figure nous pouvons remarquer que la majeure partie des
étudiants de notre études proviennent de la section pedagogique avec
enivoront 28\% ensuite vienne la section commerciale et administrative
avec 19\% , suivent sociale avec 13\%, scientifique bio-chimie avec 14
\% ensuite viennent autres differentes options avec des valeurs
inferieurs à 5\%

    \subparagraph{Attribut School}\label{attribut-school}

    cette attribue comprend les valeurs de l'ecole de provenance des nos
finaliste combiné avec l'attribue school status il joue un role
important dans notre étude.

    voyons d'abord combien des valeurs differentes il comprend:

            \begin{Verbatim}[commandchars=\\\{\}]
{\color{outcolor}Out[{\color{outcolor}13}]:} 594
\end{Verbatim}
        
    nous pouvons aisement remarquer que les eleves proviennent de 594 écoles
differentes dans l'image qui va suivre nous allons visulaise les écoles
les plus representées

    Nous pouvons remarquer aisement que le top 10 des école de provenance
est constituer de grandes écoles de la ville de Goma avec l'insititut
metanoia et le college mwanga en tete de liste avec l'institut mwanga et
metanoia en tete de liste avec 15\% chacun ensuite vienne l'institut
bakanja avec 6\% ensuite vienne maendelo, le lycée sainte ursule et
l'institut de Goma avec 6\%, 5\% et 5 \% respectivement et d'autres
éecole se partagent le reste de 50\%.

    Pour finaliser l'analyser univarrié des nos données en entrée nous
allons jetter un coup d'oeil sur la colonne Fac qui contient la faculté
choisie par l'etudiant.

            \begin{Verbatim}[commandchars=\\\{\}]
{\color{outcolor}Out[{\color{outcolor}8}]:} Faculté des Sciences Économiques et de Gestion           1549
        Faculté des Sciences et Technologies Appliquées           903
        Faculté de Droit                                          896
        Faculté de Santé et Développement Communautaires          758
        Faculté de Médecine                                       242
        Faculté de Psychologie et des Sciences de l'Éducation     227
        Faculté de Théologie                                      140
        Name: FAC, dtype: int64
\end{Verbatim}
        
    Nous allons remplacer les non des facultées par leurs abreviations
respectifs

    \begin{Verbatim}[commandchars=\\\{\}]
/Users/espyMur/Desktop/Memory-WorkingDir/memoryVenv/lib/python2.7/site-packages/pandas/core/indexing.py:141: SettingWithCopyWarning: 
A value is trying to be set on a copy of a slice from a DataFrame

See the caveats in the documentation: http://pandas.pydata.org/pandas-docs/stable/indexing.html\#indexing-view-versus-copy
  self.\_setitem\_with\_indexer(indexer, value)

    \end{Verbatim}

            \begin{Verbatim}[commandchars=\\\{\}]
{\color{outcolor}Out[{\color{outcolor}10}]:} FSEG    1549
         FSTA     903
         FD       896
         FSDC     758
         FM       242
         FPSE     227
         FT       140
         Name: FAC, dtype: int64
\end{Verbatim}
        
    \begin{center}
    \adjustimage{max size={0.9\linewidth}{0.9\paperheight}}{DataExplorationDraft1_files/DataExplorationDraft1_164_0.png}
    \end{center}
    { \hspace*{\fill} \\}
    
     Nous remarquons la distribution des valeurs pour
l'attribue faculté des étudiants: - FSGEG: 32,8\% - FSTA : 19,153\% - FD
:19\% - FSDC :16\% - FM : 5\% - FPSE : 4\% - FT :3\%

    Nous pouvons maintenant passer à l'analyse bi-varié

    \paragraph{Statistique Bivariée}\label{statistique-bivariuxe9e}

    Dans cette partie nous allons nous allons effectué une analyse bivariée
entre les attribues en entrée et les attribues en sortie , pour une
première approche nous allons faire une analyse entre la faculté choisie
et les différentes variables d'entres de notre ensemble d'apprentissage
dans la seconde approche nous essayerons de le faire la meme chose pour
le autres variables de sortie. voici le differentes combinaisons que
nous allons effectuer: 1. FAC-Diplome Province : Pour voir la relation
entre la faculté et la province d'origine de l'etudiant 2.
FAC-DIPPOURCENTAGE: Pour voir la relation entre la faculté et la
pourcentage obtenu à l'exetat 3. FAC-AGE: Pour voir la relation entre
l'age de l'etudiant et la FAC 4. FAC-DIPLOMEOPTION: Pour voir la
relation entre la faculté et l'option du diplome 5. FAC-SEXE: Pour la
relation avec le sexe des étudiants 6. FAC-SCHOOL: pour la relation
entre l'ecole de provenance la fac 7. FAC-SCOOLSTATUS : pour la relation
entre le status de l'ecole de la FAC

    FACULTE DIPLOME PROVINCE

    \begin{enumerate}
\def\labelenumi{\arabic{enumi}.}
\tightlist
\item
  Definition et Nature de la relation: Nous allons essayer de decouvrir
  la relation existante entre la province de l'ecole et le choix de ma
  section
\item
  Pour determiner le type nous allons utiliser le test de chi-carré qui
  nous permettra de trouver la force de la cette relation
\end{enumerate}

    Notre hypothèse est la suivante est la suivante : il n'yas pas de
relation etntre la province du diplome et la section coise

    Nous nous somme servie d'une classe specialisé qui contient les methodes
qui nous permettrons de conduire notre test statistique

    Prèmierement nous allons cherche la table de contigence qui nous montre
comment nos valeurs se repartissent en terme des section

            \begin{Verbatim}[commandchars=\\\{\}]
{\color{outcolor}Out[{\color{outcolor}128}]:} SCHOOLPROVINCE
          BANDUNDU              2.298197
          BAS CONGO             0.383033
          EQUATEUR              0.766066
          K OR                  0.191516
          KASAI OCCIDENTAL      0.574549
          KASAI ORIENTAL        1.915164
          KATANGA               4.596394
          KIGALI                0.383033
          KINSHASA             13.406151
          MANIEMA               7.852174
          NORD-KIVU           708.610817
          NYARUKENGE            0.191516
          ORIENTALE            15.895864
          OUEST                 0.383033
          SUD-KIVU            140.956098
          inconnu               4.596394
          Name: FSTA, dtype: float64
\end{Verbatim}
        
    Dans ces 2 tableau nous avons les differentes tables l'une c'est pour
les valleurs observées et l'autre c'est pour les valeurs attendu ainsi
avec ces valeurs nous pouvons trouver la valeur du test statistique
chi-carrée ensuite la comparée avec notre valeur critique et ensuiste
verifier si nous pouvons confirmé ou infimé notre hypothèse nulle

            \begin{Verbatim}[commandchars=\\\{\}]
{\color{outcolor}Out[{\color{outcolor}124}]:} ('It indicates that the relationship between the variables is significant at confidence of \%s',
           0.95)
\end{Verbatim}
        
            \begin{Verbatim}[commandchars=\\\{\}]
{\color{outcolor}Out[{\color{outcolor}125}]:} 113.1452701425554
\end{Verbatim}
        
    Notre valeur critique est de 113,14

    et notre valeur du test statistique est donnée par :

    lorque nous comparons ces 2 valeurs nous remarquons que notre valeur de
chi2 est superieur à notre valeur critique et tombe dans la region de
rejet de notre hypothèse nulle et ainsi celle est rejetée et ainsi nou
concluons avec 95\% de certitude que notre hypothèse alternative est
vrai: il ya une relation entre la province et la faculté en d'autre
terme ayant un nombre des étudiant venant d'une province nous pouvons
predire la faculté choisie.

    Mais cette relation est à prendre avec reserve car les données sont
inegalement repartie entre les provinces!

    Nous pouvons remarquer les résultat des nos analyser dans cette figure.

    FACULTE DIPLOMEOPTION

    nous allons refaire la meme analyse que pour le point precedant!!

            \begin{Verbatim}[commandchars=\\\{\}]
{\color{outcolor}Out[{\color{outcolor}147}]:} OPTION\_RIGHT  agrecole  agronomie  batiment  bio-chimie  \textbackslash{}
          FAC                                                       
          FD                   1          0         0          28   
          FM                   1          1         0         111   
          FPSE                 0          0         0           4   
          FSDC                 1          0         0         117   
          FSEG                 1          0         1         156   
          FSTA                 5          2         2         244   
          FT                   0          1         0           6   
          
          OPTION\_RIGHT  commerciale informatique  commmerciale et adm  construction  \textbackslash{}
          FAC                                                                         
          FD                                  12                   62             0   
          FM                                   4                    5             0   
          FPSE                                 0                    2             0   
          FSDC                                 1                   31             0   
          FSEG                                63                  724             3   
          FSTA                                11                   59            48   
          FT                                   0                   10             2   
          
          OPTION\_RIGHT  coupe couture  diet  economie     {\ldots}       machine outil  \textbackslash{}
          FAC                                             {\ldots}                       
          FD                        3     0         1     {\ldots}                   0   
          FM                        0     0         0     {\ldots}                   0   
          FPSE                      2     0         0     {\ldots}                   0   
          FSDC                     10     2         0     {\ldots}                   0   
          FSEG                      4     0         1     {\ldots}                   0   
          FSTA                      0     0         0     {\ldots}                   7   
          FT                        0     0         0     {\ldots}                   0   
          
          OPTION\_RIGHT  math-physique  mec gene  mecanique machines outils  nutr  \textbackslash{}
          FAC                                                                      
          FD                       10         3                          0     1   
          FM                       13         0                          0    14   
          FPSE                      2         0                          0     0   
          FSDC                     11         0                          0    47   
          FSEG                     42         2                          0     6   
          FSTA                    149        64                         11     1   
          FT                        4         0                          0     3   
          
          OPTION\_RIGHT  pedagogie  relations publiques  secretariat  sociale  \textbackslash{}
          FAC                                                                  
          FD                  327                    0            0      169   
          FM                   48                    0            1       27   
          FPSE                189                    0            0       16   
          FSDC                290                    0            0      187   
          FSEG                270                    1            5      184   
          FSTA                114                    0            2       51   
          FT                   90                    0            0       13   
          
          OPTION\_RIGHT  vétérinaire  
          FAC                        
          FD                      0  
          FM                      4  
          FPSE                    0  
          FSDC                    5  
          FSEG                    1  
          FSTA                    0  
          FT                      1  
          
          [7 rows x 30 columns]
\end{Verbatim}
        
            \begin{Verbatim}[commandchars=\\\{\}]
{\color{outcolor}Out[{\color{outcolor}148}]:} OPTION\_RIGHT  agrecole  agronomie  batiment  bio-chimie  \textbackslash{}
          FAC                                                       
          FD            1.710286   0.760127  0.570095  126.561188   
          FM            0.461930   0.205302  0.153977   34.182821   
          FPSE          0.433298   0.192577  0.144433   32.064051   
          FSDC          1.446872   0.643054  0.482291  107.068505   
          FSEG          2.956734   1.314104  0.985578  218.798303   
          FSTA          1.723648   0.766066  0.574549  127.549947   
          FT            0.267232   0.118770  0.089077   19.775186   
          
          OPTION\_RIGHT  commerciale informatique  commmerciale et adm  construction  \textbackslash{}
          FAC                                                                         
          FD                           17.292895           169.698409     10.071686   
          FM                            4.670626            45.833722      2.720255   
          FPSE                          4.381124            42.992789      2.551644   
          FSDC                         14.629480           143.561824      8.520467   
          FSEG                         29.895864           293.373701     17.411877   
          FSTA                         17.427996           171.024178     10.150371   
          FT                            2.702015            26.515376      1.573701   
          
          OPTION\_RIGHT  coupe couture      diet  economie     {\ldots}       machine outil  \textbackslash{}
          FAC                                                 {\ldots}                       
          FD                 3.610604  0.380064  0.380064     {\ldots}            1.330223   
          FM                 0.975186  0.102651  0.102651     {\ldots}            0.359279   
          FPSE               0.914740  0.096288  0.096288     {\ldots}            0.337010   
          FSDC               3.054507  0.321527  0.321527     {\ldots}            1.125345   
          FSEG               6.241994  0.657052  0.657052     {\ldots}            2.299682   
          FSTA               3.638812  0.383033  0.383033     {\ldots}            1.340615   
          FT                 0.564157  0.059385  0.059385     {\ldots}            0.207847   
          
          OPTION\_RIGHT  math-physique   mec gene  mecanique machines outils       nutr  \textbackslash{}
          FAC                                                                            
          FD                43.897349  13.112195                   2.090350  13.682291   
          FM                11.856204   3.541463                   0.564581   3.695440   
          FPSE              11.121315   3.321951                   0.529586   3.466384   
          FSDC              37.136373  11.092683                   1.768399  11.574973   
          FSEG              75.889502  22.668293                   3.613786  23.653871   
          FSTA              44.240297  13.214634                   2.106681  13.789183   
          FT                 6.858961   2.048780                   0.326617   2.137858   
          
          OPTION\_RIGHT   pedagogie  relations publiques  secretariat     sociale  \textbackslash{}
          FAC                                                                      
          FD            252.362248             0.190032     1.520255  122.950583   
          FM             68.160339             0.051326     0.410604   33.207635   
          FPSE           63.935525             0.048144     0.385154   31.149311   
          FSDC          213.493955             0.160764     1.286108  104.013998   
          FSEG          436.282503             0.328526     2.628208  212.556310   
          FSTA          254.333828             0.191516     1.532131  123.911135   
          FT             39.431601             0.029692     0.237540   19.211029   
          
          OPTION\_RIGHT  vétérinaire  
          FAC                        
          FD               2.090350  
          FM               0.564581  
          FPSE             0.529586  
          FSDC             1.768399  
          FSEG             3.613786  
          FSTA             2.106681  
          FT               0.326617  
          
          [7 rows x 30 columns]
\end{Verbatim}
        
            \begin{Verbatim}[commandchars=\\\{\}]
{\color{outcolor}Out[{\color{outcolor}151}]:} 4155.0246292496686
\end{Verbatim}
        
            \begin{Verbatim}[commandchars=\\\{\}]
{\color{outcolor}Out[{\color{outcolor}152}]:} ' It indicates that both categorical variable are dependent'
\end{Verbatim}
        
            \begin{Verbatim}[commandchars=\\\{\}]
{\color{outcolor}Out[{\color{outcolor}153}]:} 205.77862677980613
\end{Verbatim}
        
    sur base de ce fait on fait la meme conclusion que pour la province : il
ya une relation de dependance entre la section du diplome et la faculté
ce qui est loqique car les étudiant en genral se base sur leur option
pour chosir leur option

    \#\#\#\#\#\# FACULTE SCHOOLSTATUS

    nous allons refaire la meme analyse que pour le point precedant!!

            \begin{Verbatim}[commandchars=\\\{\}]
{\color{outcolor}Out[{\color{outcolor}223}]:} 304.71578214658456
\end{Verbatim}
        
            \begin{Verbatim}[commandchars=\\\{\}]
{\color{outcolor}Out[{\color{outcolor}224}]:} ('It indicates that the relationship between the variables is significant at confidence of \%s',
           0.99)
\end{Verbatim}
        
            \begin{Verbatim}[commandchars=\\\{\}]
{\color{outcolor}Out[{\color{outcolor}225}]:} 66.206236283993221
\end{Verbatim}
        
    sur base de ce fait on fait la meme conclusion que pour la province : il
ya une relation de dependance entre la section du diplome et la faculté
ce qui est loqique car les étudiant en genral se base sur leur option
pour chosir leur option

    FACULTE SCHOOL

            \begin{Verbatim}[commandchars=\\\{\}]
{\color{outcolor}Out[{\color{outcolor}202}]:} FAC
          FD      29
          FM       2
          FPSE     0
          FSDC     2
          FSEG    42
          FSTA     3
          FT       0
          Name: zanner, dtype: int64
\end{Verbatim}
        
            \begin{Verbatim}[commandchars=\\\{\}]
{\color{outcolor}Out[{\color{outcolor}203}]:} FAC
          FD      14.822481
          FM       4.003393
          FPSE     3.755249
          FSDC    12.539555
          FSEG    25.625027
          FSTA    14.938282
          FT       2.316013
          Name: zanner, dtype: float64
\end{Verbatim}
        
            \begin{Verbatim}[commandchars=\\\{\}]
{\color{outcolor}Out[{\color{outcolor}197}]:} 6389.4044131161027
\end{Verbatim}
        
            \begin{Verbatim}[commandchars=\\\{\}]
{\color{outcolor}Out[{\color{outcolor}198}]:} ('It indicates that the relationship between the variables is significant at confidence of \%s',
           0.95)
\end{Verbatim}
        
            \begin{Verbatim}[commandchars=\\\{\}]
{\color{outcolor}Out[{\color{outcolor}199}]:} 3697.8815995010605
\end{Verbatim}
        
    reste à verifier ....

    FACULTE SEXE

            \begin{Verbatim}[commandchars=\\\{\}]
{\color{outcolor}Out[{\color{outcolor}327}]:}           M          F
          GENDER    F    H   FAC
          FD      356  540   896
          FM      109  133   242
          FPSE    119  108   227
          FSDC    469  289   758
          FSEG    726  823  1549
          FSTA    140  763   903
          FT       25  115   140
\end{Verbatim}
        
            \begin{Verbatim}[commandchars=\\\{\}]
{\color{outcolor}Out[{\color{outcolor}264}]:} 896
\end{Verbatim}
        
            \begin{Verbatim}[commandchars=\\\{\}]
{\color{outcolor}Out[{\color{outcolor}257}]:} 4715.0
\end{Verbatim}
        
            \begin{Verbatim}[commandchars=\\\{\}]
{\color{outcolor}Out[{\color{outcolor}171}]:} 445.85875262024177
\end{Verbatim}
        
            \begin{Verbatim}[commandchars=\\\{\}]
{\color{outcolor}Out[{\color{outcolor}175}]:} ('It indicates that the relationship between the variables is significant at confidence of \%s',
           0.95)
\end{Verbatim}
        
            \begin{Verbatim}[commandchars=\\\{\}]
{\color{outcolor}Out[{\color{outcolor}173}]:} 12.591587243743977
\end{Verbatim}
        
    sur base de ce fait on fait la meme conclusion que pour la province : il
ya une relation de dependance entre le sexe et la faculté ce qui est
loqique car par example en faculté de technologie et teologie on trouve
moin des hommes que des femmes

    FACULTE - DIPLOME AGE et FACULTE DIPLOME POURCENTAGE

    comme ces une des colonnes dispose des variables continues nous allons
utiliser le test A-NOVA

    ce test nous permettra de savoir si la moyenne de l'age des étudiants
est la même pour chaque faculté: cela constituera notre hypothèse nulle
,on va cherche la probabilité p est on décidera sur base de cette valeur
! si elle es inférieur à 0.05 on rejettera l'hypothèse nulle

            \begin{Verbatim}[commandchars=\\\{\}]
{\color{outcolor}Out[{\color{outcolor}314}]:}               df        sum\_sq      mean\_sq           F         PR(>F)
          C(FAC)       6.0  14857.002658  2476.167110  135.823792  3.721673e-159
          Residual  4708.0  85830.284723    18.230732         NaN            NaN
\end{Verbatim}
        
    comme la valeur de PR est inferieur à 0.05 on peut conclure qua la
moyenne de l'age n'est pas la meme au sein de chaque faculté

            \begin{Verbatim}[commandchars=\\\{\}]
{\color{outcolor}Out[{\color{outcolor}320}]:}       IDENTIFICATION  DIPLOMPERCENTAGE        AGE
          FAC                                              
          FD       8837.433036         56.151372  24.771205
          FM      10887.719008         59.434420  21.487603
          FPSE     8417.453744         56.202099  28.224670
          FSDC     8195.007916         55.329211  25.974934
          FSEG     8410.416398         56.832564  24.323434
          FSTA     9166.437431         58.941594  23.307863
          FT       8125.250000         53.814286  31.428571
\end{Verbatim}
        
    Pour prouver le rejet de notre hypothèse nulle on peut remarquer que les
facultées de Medecine et celui de technologie on une moyenne d'age de 21
et 23 respectivement et les facultées de psycologie et celui de
theologie on une mooyenne d'age respective de 28 et 31 ans

    \paragraph{Diplome pourcentage}\label{diplome-pourcentage}

    voici comment se presente le test:

            \begin{Verbatim}[commandchars=\\\{\}]
{\color{outcolor}Out[{\color{outcolor}325}]:}               df         sum\_sq      mean\_sq          F        PR(>F)
          C(FAC)       6.0    9139.202728  1523.200455  48.757704  2.256165e-58
          Residual  4708.0  147078.865397    31.240201        NaN           NaN
\end{Verbatim}
        
    Egalement ici on rejete aussi notre hypothèse nulle qui stipulait que la
moyenne du pourcentage du diplome est la meme au seind de chaque faculté
. Pour prouver le rejet de notre hypothèse nulle on peut remarquer que
les facultés de Medecine et celui de technologie on une moyenne de
pourcentage de 59\% chacun et celui de theologie à une moyenne de 53\%

            \begin{Verbatim}[commandchars=\\\{\}]
{\color{outcolor}Out[{\color{outcolor}324}]:} 3.3032675567110044
\end{Verbatim}
        
    \begin{Verbatim}[commandchars=\\\{\}]

          File "<ipython-input-1-b7f99248bc48>", line 1
        ipython nbconvert --to pdf --template hidecode DataExplorationDraft1.ipynb
                        \^{}
    SyntaxError: invalid syntax


    \end{Verbatim}


    % Add a bibliography block to the postdoc
    
    
    
    \end{document}
