%       $Id$    
\newcommand{\monlabel}[1]{\mbox{\bfseries\textsf{#1}}\dotfill}
\newenvironment{monglossaire}%
{\begin{list}{}%
    {\renewcommand{\makelabel}{\monlabel}%
      \setlength{\labelwidth}{.3\columnwidth}%
      \setlength{\itemsep}{3cm}%
%      \setlength{\parsep}{140pt}%
      \setlength{\labelsep}{0pt}%
      \setlength{\leftmargin}{\labelwidth}
      \addtolength{\leftmargin}{\labelsep}%
    }%
}%
{\end{list}}%

\vfill 

\chapter*{Glossaire}\markboth{Glossaire}{}
\addcontentsline{toc}{chapter}{Glossaire}

\vspace*{1cm}

\begin{monglossaire}
\item [Machine Parall�le] Machine souvent utilis�e pour de lourdes
  t�ches (gros calculs scientifiques, serveur de nombreux
  utilisateurs,...). Elle est compos�e de plusieurs processeurs
  et int�gre des proc�dures sp�cialis�es permettant
  de les mettre � profit.  
\item[SGBD] Syst�me de Gestion de Bases de Donn�es. C'est l'ensemble
  des programmes qui permettent de manipuler les donn�es contenues
  dans une base de donn�es. 
\item[SQL] Structured Query Language. Langage de manipulation,
  d�finition de donn�es. Ce langage a �t� plusieurs fois normalis�. Il
  est tr�s r�pandu et largement utilis� dans la plupart des SGBD,
  surtout pour la communication entre clients et serveurs.

\item[OLAP] Online Analytical Processing. 
  

\item [Table de hachage] Table qui permet un acc�s rapide aux �l�ments
  d'une structure de donn�es (tableau, fichier). C'est une technique
  tr�s utilis�e pour l'implantation d'index dans les bases de
  donn�es. 

\item [Tableau de bord] Document de r�f�rence utilis� par les
  utilisateurs du syst�me d'information de d�cision. Il indique des
  valeurs d'indices qui d�crivent l'�volution de faits utiles au
  processus de d�cision. Par exemple, le bilan des ventes du mois, la
  progression du nombre de client...

\item[Validation crois�e] Proc�dure d'estimation de l'erreur r�elle
  d'un mod�le qui permet de se passer d'un ensemble test. La
  validation crois�e est souvent utilis�e lorsqu'on ne dispose pas de
  suffisamment d'exemples pour valider un mod�le.
  

\end{monglossaire}


%%% Local Variables: 
%%% mode: latex
%%% TeX-master: "main.tex"
%%% End: 
